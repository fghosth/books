\chapter{结论及展望}

垃圾邮件过滤技术是随着垃圾邮件在互联网中的蔓延而产生的一种新技术。垃圾邮件带来了巨大的危害,因此研究高过滤精度的垃圾邮件过滤算法有着重大的现实意义。

本文总结了垃圾邮件的定义,危害,垃圾邮件过滤技术的分类以及垃圾邮件过滤技术的研究现状,并从贝叶斯算法出发详细分析了文本表示方法,特征选择算法,电子邮件的格式标准,以及阈值对垃圾邮件过滤精度的影响。本文还分析了朴素贝叶斯算法在合法邮件误判风险以及反馈学习的优缺点,提出了两种扩展模型。

  在贝叶斯算法的研究上,本文主要做出了如下几点贡献:
  
1.	分析了传统文本分类文本的表示方式,提出了指纹特征。指纹特征相对于词语特征来说,能提取更多的文本信息。

2.	研究特征选择的一般方法,并根据贝叶斯原理,提出基于类条件分布的特征选择方法。

3.	分析电子邮件的结构,根据电子邮件的结构与一般文本的区别,提出集成加权过滤模型。

4.	分析合法邮件和非法邮件的后验概率区间,根据其分布,得知阈值的设置对邮件过滤精度有一定的影响,并依此提出阈值调整自适应算法。

5.	对朴素贝叶斯提出两种提升模型:最小风险贝叶斯,主动学习贝叶斯。最小风险贝叶斯能够减少正常邮件判为垃圾邮件的风险。主动学习贝叶斯适用于未标注样本的学习。

  在垃圾邮件过滤技术不断发展的同时,垃圾邮件制造者也在不断采用新的手法和手段制造和发送垃圾邮件。因此对于垃圾邮件过滤技术,我们仍然需要花费大量的时间和精力去研究和完善。
  
 今后,我们将在以下几个方面进行深入研究:
 
1.	 继续研究贝叶斯算法,提高其精度。

2.	把贝叶斯算法和其他方法(比如黑白名单技术等)结合起来,共同防御垃圾邮件。

3.	开发邮件服务器系统邮件客户端,并集成垃圾邮件过滤系统。