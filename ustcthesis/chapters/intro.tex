\chapter{绪论}

\section{引言}

\subsection{课题研究背景}
Internet的迅速发展,人与人的交往更加方便,电子邮件以其快捷低廉的特性逐渐成为人们信息交互的重要工具。人们用它来交流思想,传输文件,发表意见等,逐渐成为日常生活中不可缺少的通信工具。但是电子邮件在其给人们带来极大便利的同时也带来了一些负面影响,那就是我们每天收到的邮件有很大一部分是不请自来的。它们有些是商业广告,有些是政治宣传,有些是色情广告,还有一些甚至是病毒,这就是我们俗称的垃圾邮件。

根据美国nucleus  research 公司公布的数据,全球每天大约有140亿封垃圾邮件在网上传播,相当于地球上每个人每天都要收到两封以上的垃圾邮件。  

垃圾邮件给网民造成的经济损失是相当惊人的:据统计仅下载它们所花费的上网费和电话费用等费用,每年就会花掉全球网民94美元。作为垃圾邮件的发送方,价格是及其低廉的,通常是通过各种方式群发。而对于电子邮件服务提供商和用户来说,垃圾邮件却给他们带来了很大的危害和损失,而且如色情,电脑病毒,以及荷重欺诈信等造成的损失更是造成难以评估。
\subsection{贝叶斯研究简介}
贝叶斯的论文“关于几率性问题求解的理论”奠定了贝叶斯学派的基础。而后著名数学家laplace用贝叶斯理论导出的”相继律”使得贝叶斯理论受到人们的关注。但是由于当时贝叶斯方法在理论和实际的应用中还存在很多不完善的地方,因而在十九世纪并未被人接受。八十年代以后,人工智能的发展尤其是机器学习,数据挖掘等兴起,为贝叶斯理论的发展和应用提供了更为广阔的空间。 

 尽管对于贝叶斯学派哲学上的观点还是存在很多的异议,然而它的思想和方法在社会生活和生产实践中得到越来越广泛的应用确实不争的事实。尤其是近年来,贝叶斯方法以其独特的不确定知识表达形式,丰富的概率表达能力,综合先验知识的增量学习特性等成为当前数据挖掘众多方法中最为引人注目的焦点之一。
\subsection{贝叶斯垃圾邮件过滤发展史}
 1996年,Rvennie基于贝叶斯算法建立了一个用于邮件过滤的机器学习应用系统 ifile\cite{Rennieart},利用贝叶斯算法对邮件进行分类。在建立ifile系统的过程中,Rennie注意到每个用户有不同的邮件集,并且组织邮件 的方式也各不相同,因此允许用户手工调整被误判的邮件。   
 
  1998年,Sahami利用贝叶斯算法对邮件进行过滤时,注意到垃圾邮件具有不同于合法邮件的特有属性:例如,在快速致富类的垃圾邮件中,除了邮件的文本中会含有许多free  money之类的文本信息之外,还会有大量类似于 “!!!”的强调符号以及“\$”这种表征线的符号。所以Sahamiliy利用朴素贝叶斯算法滤邮件时,手工加入了这些特定任务的域信息短语以及具有垃圾邮件特征的字符到过滤器中,提高了过滤垃圾邮件的精确度;另外,他还利用一个表征损失率的门槛值来降低合法邮件的误判率。      
  
  2001年,Matthew等人开发了一个垃圾邮件过滤器MEF。MEF可以在UNIX中过 滤掉附件中有可执行程序的病毒邮件。该邮件过滤器首先对可执行程序的二进制码进行解码,然后把它与现有病毒的二进制码进行比较,利用朴素贝叶斯算法计算出它属于垃圾邮件的概率值,并据此作出决策。
 
\section{垃圾邮件的危害及当前状况}
\subsection{垃圾邮件的定义}
垃圾邮件一般指的是未经用户可,但却被强行塞入用户邮箱的电子邮件 。迄今 为止,垃圾邮件在国际上还没有统一的定义。 

      在《中国互联网协会反垃圾邮件规范》中垃圾邮件被界定为:
      
        1)  收件人事先没有提出要求或者不同意接收的广告,电子刊物以及各种形式 的宣传邮件 。  
        
        2) 收件人无法拒收的电子邮件。  
        
        3) 隐藏发件人身份,地址,标题等信息的电子邮件。
        
         4) 含有虚假的信息源,发件人,路由等噢,信息的电子邮件。  
         
         按照上述界定,上面四类邮件都属于垃圾邮件范畴。相反,我们可以称收到的其他邮件为”合法邮件”。对大多数用户,收到的垃圾邮件大部分都是没有主动订阅的广告,电子期刊登宣传品,其基本特征是”不请自来 “,带有商业目的(unsolicited commercial e-mail)或者政治目的 。实际上,垃圾邮件的判定会因而异,不同的用户对同一邮件的判定结果可能存在差异。虽然我们很难对垃圾邮件下一个确定的定义,但是绝大多数的垃圾邮件都具有以下某些特点 :
         
           1)  未授权   他们大多都是指未经请求而发送的电子邮件。 
           
            2)  商业目的  很多垃圾邮件都直接或者间接的和商业联系起来,如未经发件人请 求而发送的行业广告或者非法的电子邮件 。  
            
             3)  数目众多  垃圾邮件往往不是一封两封的发出,绝大多数为成百上千。 垃圾邮件造成的危害主要包括:  
             
             1) 增加破坏机械设备的可能(垃圾邮件可能携带危险的电脑病毒,对电脑硬盘资 料及系统造成威胁); 
             
              2)  影响电子邮箱的工作效率(大批量的垃圾邮件能使得邮箱堵塞,使得网络速度大幅下降);
              
                3) 影响与客户的正常业务联系,造成间接经济损失;  
                
                4) 增加合法邮件服务的成本(大量的垃圾邮件使得它们必须大幅度提高计算机及网络性能,以维持邮件服务器的正常运行,为此所花的成本要么自己消化,要么转嫁到用户身上); 
                
                 5)   耗费收件人的上网时间;  
                 
                 6) 对有用电子邮件的淹没(超出邮箱容量时,旧有的有用邮件被自动删除,或者混杂于大量的垃圾邮件中不易找出); 
                 
                  7) 影响接收人的身心健康(垃圾邮件中含有不健康内容的占有相当大的比例);
                  
                   8) 对人权的挑战(垃圾邮件可认为是强迫别人接收不同意接收的信息,是对人通信自由的侵犯); 
                   
                    9) 动摇了人们对互联网的信心(垃圾邮件不仅阻碍了信息业的发展,损害了人们对于网络交流的信心)。 
                    
                     10) 带来了严重的社会问题 (国内有些IP地址被国际上某些反垃圾邮件组织和网络运营商列入黑名单,甚至许多网络被国外屏蔽 ,使得合法邮件也不能正常使用 )。 
                     
                      据统计,美国每年因为垃圾邮件造成的直接或者间接损失高达10亿美元,全球的损失更高达20亿美元。
\subsection{我国垃圾邮件的当前状况}
目前我国垃圾邮件泛滥,情况十分严重。全球前十大垃圾邮件大国之中,中国仅次于美国高居垃圾邮件大国第二。中国网民收到的垃圾邮件数量占全球的十分之一,并且这个数字没五个月会翻一番。

我国目前拥有网民数接近一亿,其中绝大多数网民至少拥有一个电子邮件信箱,他们因此成为垃圾邮件的直接受害者。数据还显示,每个网民平均每天收到1.85封垃圾邮件,为处理这些垃圾邮件,每个网民每天需要花费3.65分钟。这意味着,全国网民每年会浪费点15亿小时的宝贵时间。  

网络安全界一直认为如果所有的邮件服务器不被非法利用,就可以有效地遏制垃圾邮件的传播。即使有部分组织可能会使用自己的邮件服务器发送垃圾邮件,但将会容易得被追查出来。我们国内面临的主要问题是Open Relay,特别是被国外的组织和个人利用,并且屡屡找到受害者的投诉,时有IP被列入国外反垃圾邮件组织黑名单的情况。
\section{垃圾邮件过滤常用技术}
随着垃圾邮件的泛滥,垃圾邮件过滤技术也在不断的发展,产生了许多对付垃圾邮件的方法。一些成熟的方法在客户端、服务器端等被广泛的应用,新的方法也在不断的产生。下面就介绍一下常用的垃圾邮件过滤技术。
\subsection{黑白名单技术}
黑名单(Black List)和白名单(White List)分别是己知的垃圾邮件发送者和可信任发 送者的IP地址或邮件地址。黑名单技术是最早出现的一种垃圾邮件过滤技术,一般的邮件服务器都有该功能。黑名单技术的原理是确定已知垃圾邮件制造者及其ISP的域名或IP地址、电子邮件地址,将其整理成黑名单,将黑名单部署在处理网关处,拒绝任何来自黑名单上的垃圾邮件制造者的邮件。白名单的原理是拒绝接收任何邮件,除非用户的邮件地址在白名单上。白名单提供两种使用方式:一种方法是用户阻止不在名单上的信件;另一种方式是系统邮件发送者发送信件,要求其回复,以证实确有邮件发送者其人,经过确认后将其列入白名单中。    

该技术的优点是不占用计算机资源,易于实施;缺点是需要手动维护黑白名单。由于垃圾邮件发送者经常修改和伪造他们的IP地址和邮件地址以逃避反垃圾邮件手段的检测,因此该方案在总体的垃圾邮件解决方案中仅起补充作用。

\subsection{反向域名验证}
该技术对邮件发送者的IP地址进行逆向名字解析,通过DNS查询来判断发送者 的IP与其声称的名字是否一致,来判断是否是垃圾邮件。如果反向DNS查找提供的域与邮件上的来源IP地址相符合,该邮件被接受。如果不符合,该邮件被拒绝。   

由于很多反向DNS目录未被有效建立,或无法正常建立,比如,任何”vanity”域名绝大多数情况下没有一个正确的反向DNS查找。在这种情况下,由这些域发送的邮件将被阻断,造成不可接受的高误报率。
\subsection{关键词过滤}
关键词过滤是一种基于内容检查的过滤技术,通常创建一些简单或复杂的与垃圾 邮件关联的单词表来识别和处理垃圾邮件,比如”免费”、”色情”等在垃圾邮件中经常出现的词语。该方法通过对邮件的信头、信体、附件的内容进行检查,判定是否符合过滤规则,从而判定是否为垃圾邮件。这是一种简单的内容过滤方式来处理垃圾邮件,它的基础是必须创建一个庞大的过滤关键词列表。      

这种技术缺陷很明显,过滤的能力同关键词有明显的联系,关键词列表造成漏报、错报的可能性比较大。垃圾邮件发送者经常会采用一些躲避关键词的技术,比如拆词、组词、将一些单词拼错,以图饶过词语过滤器,所以过滤关键词需要经常升级,以适应新的需要。现在的邮件群发软件做的也越来越智能了,由其自动生成和发送的垃圾邮件是随机生成的,不但能随机生成邮件的发件人、收件人和邮件主题,还能随机生成邮件的内容,使得该种技术目前应用范围日趋狭窄。
\subsection{基于规则评分的过滤技术}
这是一种集合了人工智能技术的应用技术。该技术对邮件进行规则判断。在规则 中,每条规则对应一个分数,当邮件符合某一条规则时,就给邮件增加相应的分数,分数越高,该邮件是垃圾邮件的可能性就越高,得分超过一定值时,该邮件将被分类为垃圾邮件。该技术过滤准确率可以达到90\%,但不能检测新的垃圾邮件,即漏检率高。为了能使评分有效,规则需要经常更新。
\subsection{贝叶斯过滤法}
贝叶斯算法是以著名数学家托马斯.贝叶斯(Thomas 贝叶斯)(1702-1761)命名的一 种基于概率分析的可能性推理理论,通过分析过去事件的知识,来预测未来的事件。贝叶斯过滤法对大量用户已经判定的垃圾邮件和合法邮件进行学习,根据垃圾邮件和合法邮件中相同词语及短语出现的概率对比来确定垃圾邮件的可能性。贝叶斯过滤法可以通过不断地学习来适应垃圾邮件的新规则。贝叶斯过滤法是阻断垃圾邮件最为精确的技术之一,过滤准确率可以达到99\%,但过滤准确性依赖大量的历史数据。
\section{本文研究的内容}
本文旨在对贝叶斯模型在垃圾邮件过滤中的应用进行深入研究,并针对其面临的一些关键问题提出有效的解决或改进方法,以提高贝叶斯模型在垃圾邮件过滤中的性能和可扩展性。现将本文所做的主要工作概括如下:  

(1) 贝叶斯分类模型研究   对目前贝叶斯邮件过滤器的基本原理和基本方法做了研究。  

(2) 文本表示   中文邮件内容一般表示为词语向量,英文文本的特征项则表示为由空格隔开的英语单词。对多封垃圾邮件进行相似度比较的时候,发现在同类垃圾邮件出现较高的是一些文本块短语,所以本文提出一种新的文本表示方法,用代表文本块的指纹作为特征项,研究垃圾邮件过滤的精度。  

(3) 文档特征选择算法研究与改进  特征选择是垃圾邮件过滤中一个重要的预处理环节,迄今为止,已经有多种特征选择算法被提出。常见的特征选择的方法有:信息增益法,互信息法,卡方检验法,主成分分析法等等。这些方法或者从信息论的角度或者从统计分析的角度,来找出含有信息最大或者影响显著的特征。针对贝叶斯垃圾邮件过滤,本文提出了一种基于类条件分布的特征选择方法,提取出类条件分布最不均匀的特征。 

(4) 邮件结构研究    邮件过滤是文本分类的一种,而邮件相对于传统的文本来说,又有其特殊结构。邮件的邮件头包含了很多信息,为此本文分析邮件头邮件正文的特征分布情况,并且据此提出邮件头邮件正文集成加权模型。 

(5)阈值对垃圾邮件过滤精度的影响   概率阈值对贝叶斯过滤器垃圾邮件过滤精度有着直接的重要影响,不同的邮件集有自己的最佳阈值,对同一个邮件集,阈值取值对过滤精度也有影响。因此本文研究阈值问题,并且提出一种阈值调整算法。  

(6)贝叶斯过滤器的扩展模型    对朴素贝叶斯过滤器进行扩展,实现了最小风险贝叶斯和主动学习贝叶斯,并且 对最小风险贝叶斯和主动学习贝叶斯应用的条件进行了分析。


%\footnote{}
