\chapter{主要源码}

\begin{lstlisting}[language=python]
<strong>'''''******************************* 
  Copyright (C), 2015-2020, cyp Tech. Co., Ltd. 
  FileName:           朴素贝叶斯实例.cpp 
  Author:             cyp      
  Version :           v1.0           
  Date:               2015-10-21 上午10:38 
  Description:        利用朴素贝叶斯原理对邮件进行分类      
  Function List:       
    1.  createVocalList:计算输入各集的并集,且每个元素是unique 
    2.  setofWords2Vec:  运用词集方式将数据预处理 
    3.  bagofWords2Vec:  运用词袋方式将数据预处理 
    4.  trainNB0:       训练数据 
    5.  classifyNB:      对email进行分类 
    6.  test             相当于主函数,训练模型并测试 
********************************************'''  
  
  
'''''*************************************** 
需要导入的模块 
********************************************''  
import random  
import math  
import numpy as np  
import os  
  
'''''********************************************* 
  Function:        createVocalList 
  Description:     计算输入各集的并集,且每个元素是unique 
  Calls:            
  Called By:       test 
  Input:           dataSet输入的数据集,及单词向量集 
  Return:          返回输入数据集的并集,返回数据的数据类型是list 
                   表示出现过的单词 
************************************************'''  
def createVocalList(dataSet):  
    vocalSet = set([])                  #创建一个空集  
    for data in dataSet:  
        vocalSet = vocalSet | set(data) 
        #利用set的特性求两个集合的并集  
    return list(vocalSet)               
    #以list的方式返回结果  
      
      
'''''******************************************** 
  Function:        setofWords2Vec 
  Description:     判断的向量的单词是否在词库中是否出现,如果出现 
                   则标记为1,否则标记为0 
  Calls:            
  Called By:       test 
  Input: 
                   vocabList: 
                   词库,曾经出现过的单词集合,即createVocalList 
                   inputSet: 
                   测试的单词向量 
  Return:          returnVec是一个同vocalList同大小的数据,
如果inputset单词出现在vocalList则将returnVec对应的位置置1,
否则为0 
************************************************'''  
def setofWords2Vec(vocalList,inputset):  
    returnVec = [0]*len(vocalList)         
    #同vocalList同大小的list,开始假设每个单词都没出现,即全为0  
    for word in inputset:                                          
        if word in vocalList:        
        #如果输入的单词出现在词库中,则置1     
            returnVec[vocalList.index(word)] = 1  
        else:                        
        #如果没出现在词库中,默认为0,并输出  
      print("the word:%s is not in my Vocabulary!"%word)  
    return returnVec                 #返回结果  
'''''********************************************
  Function:        bagofWords2Vec 
  Description:     统计的单词是否在词库中出现频率 
                  
  Calls:            
  Called By:       test 
  Input: 
                   vocabList: 
              词库,曾经出现过的单词集合,即createVocalList 
                   inputSet: 
                   测试的单词向量 
  Return:     returnVec是一个同vocalList同大小的数据,对应位置 
                    表示该单词出现的个数 
***********************************************'''  
def bagofWords2Vec(vocalList,inputset):  
    returnVec = [0]*len(vocalList)         
    #同vocalList同大小的list,开始假设每个单词都没出现,即全为0  
    for word in inputset:                                          
        if word in vocalList:  #如果输入的单词出现在词库中,则加1     
            returnVec[vocalList.index(word)] += 1  
    return returnVec                 #返回结果  
'''''****************************************** 
  Function:        trainNB0 
Description: 统计侮辱性email出现概率,侮辱性邮件各单词出现的概率,
               非侮辱性email各单词出现概率 
  Calls:            
  Called By:       test 
  Input: 
                   trainDataSet:参与训练的数据集,
                   即setofWords2Vec返回数据的集合 
                   trainLabels:训练数据的标签 
  Return:           
                    pShame:侮辱性email 
                    p0Vect:非侮辱性email中单词出现的概率 
                    p1Vect:侮辱性email中单词出现的概率 
********************************************'''  
def trainNB0(trainDataSet,trainLabels):  
    numTrains = len(trainDataSet)     #训练数据的组数  
    numWords = len(trainDataSet[0])   #每组训练数据的大小  
    pShame = sum(trainLabels)/float(numTrains)  
#标签中1表示侮辱,0表示非侮辱,故上述统计侮辱性email出现概率  
#zeros(numWords)表示1*numWords数组,非numWords*numWords数组  
#p0Num = np.zeros(numWords)  #存储统计非侮辱性邮件各单词出现频率  
#p1Num = np.zeros(numWords)   #存储统计侮辱性邮件各单词出现频率  
#p0SumWords = 0.0                  #非侮辱性邮件中单词总数  
#p1SumWords = 0.0                  #侮辱性邮件中单词总数  
#为了防止与0相乘,故初始化为1,因为两者的初始化一样,所以不影响结果  
p0Num = np.ones(numWords)   #存储统计非侮辱性邮件各单词出现频率  
p1Num = np.ones(numWords)     #存储统计侮辱性邮件各单词出现频率  
p0SumWords = 2.0                    #非侮辱性邮件中单词总数  
p1SumWords = 2.0                    #侮辱性邮件中单词总数  
    for i in range(numTrains):  
        if trainLabels[i]==1:         #如果为侮辱性email  
            p1Num += trainDataSet[i]  #统计非侮辱性邮件各单词  
        else:  
            p0Num += trainDataSet[i]  #统计侮辱性邮件各单词  
p0SumWords = sum(p0Num)           #计算非侮辱性邮件中单词总数  
p1SumWords = sum(p1Num)           #计算侮辱性邮件中单词总数  
p0Vect = p0Num/p0SumWords         #非侮辱性邮件中单词出现的概率  
p1Vect = p1Num/p1SumWords         #侮辱性邮件中单词出现的概率  
return pShame,p0Vect,p1Vect  
      
'''''************************************* 
  Function:        classifyNB 
  Description:     对email进行分类 
  Calls:            
  Called By:       test 
  Input: 
              vec2Classify:要分类的数据 
              pShame:侮辱性email 
               p0Vect:非侮辱性email中单词出现的概率 
             p1Vect:侮辱性email中单词出现的概率 
  Return:           
      分类的结果,1表示侮辱性email,0表示非侮辱性email 
********************************************'''  
def classifyNB(vec2Classify,p0Vect,p1Vect,pShame):  
    #由于小数相乘,会有很大误差,故先求对数再相乘  
    temp0 = vec2Classify*p0Vect   
    temp1 = vec2Classify*p1Vect  
    temp00 = []   
    temp11 = []  
    #分步求对数,因为log不能处理array,list  
    for x in temp0:  
        if x>0:  
            temp00.append(math.log(x))   
        else:  
            temp00.append(0)  
    for x in temp1:  
        if x>0:  
            temp11.append(math.log(x))   
        else:  
            temp11.append(0)  
p0 = sum(temp00)+math.log(1-pShame)#属于非侮辱性email概率  
p1 = sum(temp11)+math.log(pShame) #属于侮辱性email概率  
if p1 > p0: #属于侮辱性email概率大于属于非侮辱性email概率  
        return 1  
    else:  
        return 0  
  
'''''********************************************* 
  Function:        text2VecOfWords 
  Description:     从email的string中提取单词 
  Calls:            
  Called By:       test 
  Input: 
                   string: email字符串 
  Return:           
                   单词向量 
**********************************************'''  
def text2VecOfWords(string):  
    import re        #正则表达式工具  
    listOfWolds = re.split(r"\W*",string)  
    #分割数据,其分隔符是除单词,数字外任意的字符串   
return [word.lower() 
for word in listOfWolds if len(word)>2]  
    #单词不区分大小写,及全部转换为(小写),滤除没用短字符串  
  
'''''*****************************************
  Function:        test 
  Description:     将数据部分用来训练模型,部分用来测试 
  Calls:           text2VecOfWords 
                   createVocalList 
                   trainNB0 
                   classifyNB 
  Called By:       main 
  Input: 
  Return:           
******************************************'''  
def test():  
    emailList = []               #存放每个邮件的单词向量  
    emailLabel = []              #存放邮件对应的标签  
    cwd = "C:\\Users\\cyp\\Documents\\
    sourcecode\\machinelearning\\my__code\\chapter4\\"  
    for i in range(1,26):  
        #读取非侮辱邮件,并生成单词向量  
  wordList = text2VecOfWords(open(cwd+r"email\ham\%d.txt"%i
        ,encoding='Shift_JIS').read())  
        emailList.append(wordList)#将单词向量存放到emailList中  
        emailLabel.append(0)      #存放对应的标签  
        wordList = text2VecOfWords(open(
  cwd+r"email\spam\%d.txt"%i,encoding='Shift_JIS').read())  
        #读取侮辱邮件,并生成单词向量  
        emailList.append(wordList)#将单词向量存放到emailList中  
        emailLabel.append(1)      #存放对应的标签  
    vocabList = 
    createVocalList(emailList)#由所有的单词向量生成词库  
    trainSet = [i for i in range(50)]   #产生0-49的50个数字  
    testIndex = []                    #存放测试数据的下标  
    for i in range(10):               #从[0-49]中随机选取10个数  
        randIndex = int(random.uniform(0,len(trainSet)))  
                            #随机生成一个trainSet的下标  
        testIndex.append(trainSet[randIndex])
        #提取对应的数据作为训练数据  
        del(trainSet[randIndex])  
        #删除trainSet对应的值,以免下次再选中  
    trainDataSet = []        #存放训练数据(用于词集方法)  
    trainLabels = []         #存放训练数据标签(用于词集方法)  
    trainDataSet1 = []         #存放训练数据(用于词袋方法)  
    trainLabels1 = []          #存放训练数据标签(用于词袋方法)  
    for index in trainSet:     #trainSet剩余值为训练数据的下标  
        trainDataSet.append(
        setofWords2Vec(vocabList,emailList[index]))  
        #提取训练数据  
        trainLabels.append(emailLabel[index])  
        #提取训练数据标签  
        trainDataSet1.append(
        bagofWords2Vec(vocabList,emailList[index]))  
        #提取训练数据  
        trainLabels1.append(emailLabel[index])  
        #提取训练数据标签  
  pShame,p0Vect,p1Vect = trainNB0(trainDataSet,trainLabels)
    #开始训练  
pShame1,p0Vect1,p1Vect1 = 
trainNB0(trainDataSet1,trainLabels1)
    #开始训练  
    errorCount = 0.0          #统计测试时分类错误的数据的个数  
    errorCount1= 0.0          #统计测试时分类错误的数据的个数  
    for index in testIndex:  
      worldVec = setofWords2Vec(vocabList,emailList[index])
        #数据预处理  
        #进行分类,如果分类不正确,错误个位数加1  
        if classifyNB(worldVec,p0Vect,p1Vect,pShame) 
        != emailLabel[index]:  
            errorCount += 1  
      worldVec = bagofWords2Vec(vocabList,emailList[index])
        #数据预处理  
        #进行分类,如果分类不正确,错误个位数加1  
        if classifyNB(worldVec,p0Vect1,p1Vect1,pShame1) 
        != emailLabel[index]:  
            errorCount1 += 1  
    #输出分类错误率  
    print("As to set,the error rate is :",
    float(errorCount)/len(testIndex))  
    print("AS to bag,the error rate is :",
    float(errorCount1)/len(testIndex))  
      
'''''************************************ 
  Function:       main 
  Description:    运行test函数 
  Calls:          test 
  Called By:        
  Input: 
  Return:           
*******************************************'''  
if __name__=="__main__":  
    test()</strong>  

\end{lstlisting}
